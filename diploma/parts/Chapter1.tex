\section{Постановка задачи}
\label{sec:Chapter1} \index{Chapter1}

% Необходимо формально изложить суть задачи в данной секции, предоставив такие
% ясные и точные описания, которые позволят в последующем оценить, насколько
% разработанное решение соответствует поставленной задаче. Текст главы должен
% следовать структуре технического задания, включая как описание самой задачи,
% так и набор требований к ее решению.

Ручное исправление состояний гонок в схемах сборки --- трудный процесс. Цель этой работы --- предоставить решение, которое бы позволило его упростить. Для этого предлагается разработать автоматический инструмент --- санитайзер для параллельных сборок. Он должен отвечать следующим требованиям:

\begin{itemize}
	\item Инструмент должен обнаруживать все гонки, связанные с ошибками в схеме сборки.
	\item Алгоритм поиска состояний гонок не должен носить вероятностный характер. Последовательные запуски инструмента на одном и том же проекте должны сообщать об одних и тех же гонках. 
	\item Инструмент должен быть легко встраиваем в существующие проекты, не должен требовать значительных изменений в проект и не должен вмешиваться в процесс сборки. 
	\item  Не должны требоваться многократные пересборки проекта или отключение многопоточности (\texttt{-j1}), не должен значительно замедляться сам процесс сборки проекта.
\end{itemize}

Основная задача, из которой следуют все вышеперечисленные пункты --- сократить время, требуемое разработчику для поиска гонок.