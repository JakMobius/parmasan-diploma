\section{Заключение}
\label{sec:Chapter5} \index{Chapter5}

% Здесь надо перечислить все результаты, полученные в ходе работы. Из текста
% должно быть понятно, в какой мере решена поставленная задача.

В рамках этой работы была рассмотрена проблема состояний гонок при параллельной сборке. Был проведён анализ проблемы и приведены недостатки существующих решений. Для разработки нового решения была представлена классификация распространённых типов гонок и разработаны методики для их автоматического обнаружения. Разработан метод критических доступов и доказана его корректность применительно к обнаружению гонок двух типов. В практической части представлена архитектура программной реализации представленных алгоритмов.

Согласно гистограмме на рис. \ref{fig:testing-results}, санитайзер не является полноценной заменой режима \texttt{{-}{-}shuffle} утилиты Make. Некоторые существующие в проектах гонки не обнаруживаются санитайзером. Несмотря на корректность представленных алгоритмов, представленная в работе классификация не покрывает все возможные гонки.

Существует по крайней мере ещё один тип гонок, который не может быть обнаружен ни одним из представленных алгоритмов --- гонка между созданием и использованием символической ссылки. Известно, что некоторые из пропущенных санитайзером гонок относятся именно к этому типу. Составить алгоритм, который бы позволил производить нужные проверки для обнаружения таких гонок за допустимое время, к текущему времени не удалось. Даже с добавлением такого алгоритма в санитайзер нельзя было бы гарантировать полное покрытие всех возможных типов гонок.

Несмотря на это, в сравнении с режимом \texttt{{-}{-}shuffle} утилиты Make, разработанный инструмент в большей степени отвечает требованиям, сформулированным в главе \ref{sec:Chapter1}:

\begin{itemize}
    \item Согласно данным тестирования, в среднем санитайзер обнаруживает больше гонок, чем позволяет Make \texttt{{-}{-}shuffle};
    \item В отличие от существующего решения, алгоритмы поиска, используемые санитайзером, не носят вероятностный характер;
    \item Инструмент может быть встроен в любой проект путём использования модифицированного Make. Согласно результатам тестирования, санитайзер замедляет сборку в среднем на 12\% вне зависимости от объёма проекта. Это позволяет использовать санитайзер на проектах любого масштаба;
    \item В сравнении с Make \texttt{{-}{-}shuffle}, санитайзер требует произвести лишь единоразовую сборку проекта. Отладка гонок может производиться отложенно с использованием собранной трассы, что значительно сокращает время ожидания.
\end{itemize}

Дальнейшее улучшение санитайзера включает в себя следующие направления:

\begin{itemize}
    \item Расширение классификации гонок и разработка алгоритмов для их обнаружения (например, гонка между созданием и использованием символической ссылки);
    \item Поддержка других систем сборки, таких как Ninja;
    \item Внедрение санитайзера в существующие build-боты и CI/CD системы;
    \item Добавление протокола, подобного Chrome DevTools Protocol, для подключения внешних отладчиков;
\end{itemize}

Актуальные версии санитайзера и модифицированного Remake доступны на репозиториях GitHub:

\begin{itemize}
    \item \url{https://github.com/ispras/parmasan}
    \item \url{https://github.com/ispras/parmasan-remake}
\end{itemize}
