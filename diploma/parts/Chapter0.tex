\section{Глоссарий}
\label{sec:Glossary} \index{Glossary}

\begin{itemize}
    \item Санитайзер --- инструмент для обнаружения проблем и уязвимостей в коде программ.
    \item Makefile --- файл, задающий схему сборки проекта или модуля.
    \item Цель сборки --- элемент схемы сборки, имеющий свой рецепт и список пререквизитов.
    \item Пререквизит --- файл или цель, от которой зависит другая цель.
    \item Index node (inode) --- структура данных, владеющая содержимым файла в файловой системе. Номер inode (или серийный номер файла, см. опр. 3.176 в стандарте POSIX~\cite{8277153}) --- уникальный идентификатор файла в файловой системе.
    \item Directory entry --- опр. 3.130 в стандарте POSIX~\cite{8277153}. Структура данных, соотносящая путь в файловой системе с inode.
    \item Glob--выражение --- шаблон для выбора нескольких имен файлов или каталогов, например, \texttt{*.txt}. Символ \texttt{*} в шаблоне может быть заменён на произвольную строку, \texttt{?} --- на любой символ. Glob--выражение может быть представлено регулярным выражением.
    \item Регулярный язык --- множество слов, принимаемых некоторым регулярным выражением.
    \item МПДКА --- минимальный полный детерминированный конечный автомат. Структура данных, позволяющая определить, относится ли слово к некоторому регулярному языку за время $O(n)$, где $n$ --- длина слова.
\end{itemize}

\section{Введение}
\label{sec:Chapter0} \index{Chapter0}

% В этой части надо описать предметную область, задачу из которой вы будете
% решать, объяснить её актуальность (почему надо что-то делать сейчас?). Здесь же
% стоит ввести определения понятий, которые вам понадобятся в постановке задачи.

Состояние гонки~--- это ситуация, при которой поведение программы зависит от относительного порядка выполнения двух или более параллельных операций, и может меняться в зависимости от последовательности их выполнения. Это приводит к непредсказуемому поведению программы, и обусловлено, как правило, отсутствием синхронизации между потоками.

При рассмотрении проблематики состояний гонки в основном фокусируются на языках программирования прикладного уровня, таких как C\texttt{++} или Java. Однако, такие проблемы также могут возникать в процессе сборки программного обеспечения, где примитивами синхронизации выступают зависимости между целями сборки. Отсутствие необходимой зависимости может привести к состоянию гонки, аналогично отсутствующей синхронизации между процессами.

\begin{figure}[H]
	\centering
    \begin{tikzpicture}
        \node[style=draw, minimum width=1.5cm] (libcpp) at (2,1.5) {lib.cpp};
        \node[style=draw, minimum width=1.5cm] (libo) at (4,1.5) {lib.o};
        \node[style=draw, minimum width=1.5cm] (liba) at (6,1.5) {lib.a};

        \node[style=draw, minimum width=3cm] (appcpp) at (2.75,0) {app.cpp};
        \node[style=draw, minimum width=3cm] (appo) at (6.25,0) {app.o};
        \node[style=draw, minimum width=3cm] (app) at (9.75,0) {app};

        \node[text width=5cm] at (10.5,1.1) {
            \texttt{app} зависит от \texttt{lib.a}, но \textbf{не дожидается сборки}
        };

        % liba -> app
        \draw[gray, dashed, line width=1pt, shorten >=2pt, shorten <=2pt, ->]
        (liba) to[out=0, in=180] ($(app.west) + (0, 0.12)$);

        % Common
        \node[minimum width=16cm, minimum height=3.5cm] (sizer) at (6.5,0.3) {};
        \node at (0,1.5) {Поток 1:};
        \node at (0,0) {Поток 2:};

        % appo -> app
        \draw[line width=1pt, shorten >=2pt, shorten <=2pt, ->]
        ($(appo.east) + (0,-0.12)$) -- ($(app.west) + (0,-0.12)$);

        % Time text and arrow
        \node[font=\itshape] (time) at (11.4, -1.16) {$\tau$};
        \draw[line width=0.7pt, ->] (0, -1.16) to (11, -1.16);

        \draw[line width=1pt, shorten >=2pt, shorten <=2pt, ->] (libcpp) -- (libo);
        \draw[line width=1pt, shorten >=2pt, shorten <=2pt, ->] (libo) -- (liba);
        \draw[line width=1pt, shorten >=2pt, shorten <=2pt, ->] (appcpp) -- (appo);
    \end{tikzpicture}
    \caption{Процесс сборки проекта с состоянияем гонки в схеме сборки}
\end{figure}

Выше изображен процесс сборки проекта. В нём исходный код приложения может собираться параллельно с библиотекой, которую он использует. Это позволяет ускорить сборку всего проекта. Однако в этой схеме не указано, что перед компоновкой всего приложения необходимо дождаться, пока библиотека будет готова.

На рисунке сверху это не приводит к ошибке, поскольку библиотека сама собой успела собраться быстрее, чем она потребовалась. Однако, это не всегда может быть так. Выход из строя секторов диска, расширение самой библиотеки и множество других непредсказуемых причин могут привести к увеличению времени сборки библиотеки.

\begin{figure}[H]
	\centering
    \begin{tikzpicture}
        \node[style=draw, minimum width=2.2cm] (libcpp) at (2.35,1.5) {lib.cpp};
        \node[style=draw, minimum width=2.2cm] (libo) at (5.2,1.5) {lib.o};
        \node[style=draw, minimum width=2.2cm] (liba) at (8.05,1.5) {lib.a};

        \node[style=draw, minimum width=2cm] (appcpp) at (2.25,0) {app.cpp};
        \node[style=draw, minimum width=2cm] (appo) at (4.75,0) {app.o};
        \node[style=draw, minimum width=2cm] (app) at (7.25,0) {app};
        \node[style=draw, minimum width=2cm] (ошибка) at (9.75,0) {ошибка};

        \node[style=draw, dashed, minimum width=4cm] (error) at (3.5,0.75) {lib.a еще не готова};

        \draw[dashed, line width=1pt, shorten >=2pt, shorten <=2pt, ->]
        (error) to[out=0, in=180] ($(app.west) + (0, 0.12)$);

        \draw[line width=1pt, shorten >=2pt, shorten <=2pt, ->] (app) -- (ошибка);

        % Common
        \node[minimum width=16cm, minimum height=3.5cm] (sizer) at (6.5,0.3) {};
        \node at (0,1.5) {Поток 1:};
        \node at (0,0) {Поток 2:};

        % appo -> app
        \draw[line width=1pt, shorten >=2pt, shorten <=2pt, ->]
        ($(appo.east) + (0,-0.12)$) -- ($(app.west) + (0,-0.12)$);

        % Time text and arrow
        \node[font=\itshape] (time) at (11.4, -1.16) {$\tau$};
        \draw[line width=0.7pt, ->] (0, -1.16) to (11, -1.16);

        \draw[line width=1pt, shorten >=2pt, shorten <=2pt, ->] (libcpp) -- (libo);
        \draw[line width=1pt, shorten >=2pt, shorten <=2pt, ->] (libo) -- (liba);
        \draw[line width=1pt, shorten >=2pt, shorten <=2pt, ->] (appcpp) -- (appo);
    \end{tikzpicture}
    \caption{Процесс сборки проекта с состоянияем гонки в схеме сборки}
\end{figure}

Такие ошибки можно исправить тремя способами:

\begin{enumerate}
    \item Найти и добавить недостающую зависимость;
    \item Перезапустить сборку, если ошибка редко воспроизводится;
    \item Отключить многопоточность (указать \texttt{-j1}). Make собирает пререквизиты в таком же порядке, в котором они указаны в Makefile, поэтому отключение многопоточности исключит вероятность возникновения гонок. В некоторых проектах такое решение из временного становится постоянным, например, как в проекте \texttt{netpbm}. Для больших проектов такое решение неприменимо, поскольку в однопоточном режиме сборка может занимать в несколько раз больше времени.
\end{enumerate}

\begin{figure}[H]
	\centering
    \begin{tikzpicture}
        % Frame 3
        \node[style=draw, minimum width=2.2cm] (libcpp) at (2.35,1.5) {lib.cpp};
        \node[style=draw, minimum width=2.2cm] (libo) at (5.2,1.5) {lib.o};
        \node[style=draw, minimum width=2.2cm] (liba) at (8.05,1.5) {lib.a};

        \node[style=draw, minimum width=2cm] (appcpp) at (2.25,0) {app.cpp};
        \node[style=draw, minimum width=2cm] (appo) at (4.75,0) {app.o};
        \node[style=draw, minimum width=2cm] (app) at (10.8,0) {app};

        \node[text width=4.5cm] at (12.5,1.1) {
            \small{Поток 2 ждёт сборки \texttt{lib.a} перед сборкой \texttt{app}}
        };

        % liba -> app
        \draw[line width=1pt, shorten >=2pt, shorten <=2pt, ->]
        (liba) to[out=0, in=180] ($(app.west) + (0, 0.12)$);

        % Common
        \node[minimum width=16cm, minimum height=3.5cm] (sizer) at (6.5,0.3) {};
        \node at (0,1.5) {Поток 1:};
        \node at (0,0) {Поток 2:};

        % appo -> app
        \draw[line width=1pt, shorten >=2pt, shorten <=2pt, ->]
        ($(appo.east) + (0,-0.12)$) -- ($(app.west) + (0,-0.12)$);

        % Time text and arrow
        \node[font=\itshape] (time) at (11.4, -1.16) {$\tau$};
        \draw[line width=0.7pt, ->] (0, -1.16) to (11, -1.16);

        \draw[line width=1pt, shorten >=2pt, shorten <=2pt, ->] (libcpp) -- (libo);
        \draw[line width=1pt, shorten >=2pt, shorten <=2pt, ->] (libo) -- (liba);
        \draw[line width=1pt, shorten >=2pt, shorten <=2pt, ->] (appcpp) -- (appo);
    \end{tikzpicture}
    \caption{Исправленная схема сборки без состояния гонки}
\end{figure}

Настоящие схемы сборки, как правило, выглядят значительно сложнее, и найти в них недостающую зависимость становится трудно. В связи с этим, такие проблемы в проектах могут долго оставаться неисправленными. Подтверждение этому можно найти на форуме Gentoo, где перечислены открытые обсуждения, связанные с ошибками при параллельной сборке пакетов для этой системы~\cite{gentoo-bugzilla}.

Опасность этих гонок заключается в том, что оставаясь скрытыми, они могут проявляться непредсказуемым образом. Часто при наличии такой проблемы в схеме возникает спонтанная ошибка при сборке, которая исчезает при повторной попытке собрать проект. Существует и более опасный сценарий, при котором такая ошибка может приводить к скрытым проблемам. Например, к некорректно собранным файлам локализации или к уязвимости в распространяемом исполняемом файле.
