%% Работа с русским языком
\usepackage{cmap}			 % поиск в PDF
\usepackage{mathtext} 		 % русские буквы в формулах
\usepackage[T2A]{fontenc}	 % кодировка
\usepackage[utf8]{inputenc}	 % кодировка исходного текста
\usepackage[russian]{babel}	 % локализация и переносы

%% Пакеты для работы с математикой
\usepackage{amsmath,amsfonts,amssymb,amsthm,mathtools}
\usepackage{icomma}

%% Нумерация формул (опционально)
%\mathtoolsset{showonlyrefs=true} % показывать номера только у тех формул, на которые есть \eqref{} в тексте.
%\usepackage{leqno}               % нумерация формул слева

%% Шрифты
\usepackage{euscript}	 % шрифт "Евклид"
\usepackage{mathrsfs}    % красивый мат. шрифт

\usepackage{tikz}
\usetikzlibrary{shapes.geometric}
\usetikzlibrary{calc}

\usepackage{gnuplot-lua-tikz}
\usepackage{dcolumn}

%% Некоторые полезные макросы для дебага (в случае недоверия авторам шаблона)
\makeatletter
\newcommand\thefontsize{The current font size is: \f@size pt} % пример: \section{\thefontsize}
\makeatother

%% Настройка размеров шрифтов
\makeatletter
\renewcommand\Huge{\@setfontsize\Huge{14pt}{1.5}}
\renewcommand\huge{\@setfontsize\huge{14pt}{1.5}}
\renewcommand\Large{\@setfontsize\Large{14pt}{1.5}}
\renewcommand\large{\@setfontsize\large{12pt}{1.5}}
\makeatother

%% Поля (геометрия страницы)
\usepackage[left=3cm,right=1.5cm,top=2cm,bottom=2cm,bindingoffset=0cm]{geometry}

%% Русские списки
\usepackage{enumitem}
\makeatletter
\AddEnumerateCounter{\asbuk}{\russian@alph}{щ}
\makeatother

%% Работа с картинками
\usepackage{caption}
\captionsetup{justification=centering} % центрирование подписей к картинкам
\usepackage{graphicx}                  % вставки рисунков
\graphicspath{{images/}{images2/}}     % папки с картинками
\setlength\fboxsep{3pt}                % отступ рамки \fbox{} от рисунка
\setlength\fboxrule{1pt}               % толщина линий рамки \fbox{}
\usepackage{wrapfig}                   % обтекание рисунков и таблиц текстом

%% Работа с таблицами
\usepackage{array,tabularx,tabulary,booktabs} % дополнительная работа с таблицами
\usepackage{longtable}                        % длинные таблицы
\usepackage{multirow}                         % слияние строк в таблице

%% Красная строка
\setlength{\parindent}{2em}

%% Интервалы
\linespread{1}
\usepackage{multirow}

%% TikZ
\usepackage{tikz}
\usetikzlibrary{graphs,graphs.standard}

%% Верхний колонтитул
\usepackage{fancyhdr}
\pagestyle{fancy}

%% Перенос знаков в формулах (по Львовскому)
\newcommand*{\hm}[1]{#1\nobreak\discretionary{}{\hbox{$\mathsurround=0pt #1$}}{}}

%% Дополнительно
\usepackage{float}   % добавляет возможность работы с командой [H] которая улучшает расположение на странице
\usepackage{gensymb} % красивые градусы
\usepackage{caption} % пакет для подписей к рисункам, в частности, для работы caption*
\usepackage{listings} % пакет для листингов с кодом
\lstset{              % настройки для лисингов с кодом
basicstyle=\small\ttfamily,
columns=fixed,
basewidth=0.5em,
breaklines=true,
tabsize=8
}

% Hyperref (для ссылок внутри  pdf)
\usepackage[unicode, pdftex]{hyperref}

% Отступ перед первым абзацем в каждом разделе
\usepackage{indentfirst}
