\begin{tikzpicture}
    \def \Start{-4.97}
    \def \SymWidth{0.2115}
    \def \LineStart{\Start + \SymWidth * \StartChar}%
    \def \LineEnd{\Start + \SymWidth * \EndChar}%
    \def \TextPos{\Start + \SymWidth * \StartChar + \TextWidth / 2 - 0.1}%
    \newcommand{\DrawText}[2]{
        \node[align=flush left, font=\footnotesize, minimum width=\TextWidth cm, text width=\TextWidth cm] at (\TextPos, #1) {#2};
    }
    \newcommand{\DrawLine}[1] {
        \draw[line width=0.7pt] (\LineStart, #1) -- (\LineStart, -0.25) -- (\LineEnd, -0.25);
    }

    \node at (0, 0) {\texttt{ASYNC READ 16 /lib64/libc.so.6 1005 15 646356 7}};

    % ASYNC:
    \def \StartChar{0} \def \EndChar{5} \def \TextWidth{4}
    \DrawLine{-4.5}
    \DrawText{-5.1}{Код возврата системного вызова \texttt{open}}.

    % READ:
    \def \StartChar{6} \def \EndChar{10} \def \TextWidth{5}
    \DrawLine{-3}
    \DrawText{-3.8}{Название события. В данном случае~--- \texttt{READ}~--- чтение файла.}.

    % 16:
    \def \StartChar{11} \def \EndChar{13} \def \TextWidth{4}
    \DrawLine{-2}
    \DrawText{-2.6}{Длина строки, следующей далее.}

    % /lib64/libc.so.6:
    \def \StartChar{14} \def \EndChar{30} \def \TextWidth{3}
    \DrawLine{-0.5}
    \DrawText{-1.3}{Путь, по которому процесс обратился к файлу.}

    % 1005:
    \def \StartChar{31} \def \EndChar{35} \def \TextWidth{6}
    \DrawLine{-4}
    \DrawText{-4.6}{pid процесса, производящего доступ к файлу.};

    % 15:
    \def \StartChar{36} \def \EndChar{38} \def \TextWidth{4}
    \DrawLine{-3}
    \DrawText{-3.6}{Device number файловой системы.};

    % 646356:
    \def \StartChar{39} \def \EndChar{45} \def \TextWidth{4}
    \DrawLine{-2}
    \DrawText{-2.6}{Номер inode прочитанного файла.};

    % 7:
    \def \StartChar{46} \def \EndChar{47} \def \TextWidth{3}
    \DrawLine{-0.5}
    \DrawText{-1.3}{Код возврата системного вызова \texttt{open}.};
\end{tikzpicture}