\begin{tikzpicture}
    % symbol width = 10.2 / 47 = 0.217
    \node at (0, 0) {\texttt{ASYNC READ 16 /lib64/libc.so.6 1005 15 646356 7}};

    % ASYNC:
    % -5.1 + 0.217 * 0 = -5.1
    % -5.1 + 0.217 * 5 = -4.015
    \draw[line width=0.7pt] (-5.1, -0.25) -- (-4.0375, -0.25);
    \draw[line width=0.7pt] (-5.1, -0.25) -- (-5.1, -4.5);
    \node[align=flush left, font=\footnotesize, minimum width=4cm, text width=4cm] at (-3.2, -5.1) {Флаг синхронности сообщения.};


    % READ:
    % -5.1 + 0.217 * 6 = -3.798
    % -5.1 + 0.217 * 10 = -2.93
    \draw[line width=0.7pt] (-3.798, -0.25) -- (-2.93, -0.25);
    \draw[line width=0.7pt] (-3.798, -0.25) -- (-3.798, -3);
    \node[align=flush left, font=\footnotesize, minimum width=5cm, text width=5cm] at (-1.398, -3.8) {Название события. В данном случае~--- \texttt{READ}~--- чтение файла.};

    % 16:
    % -5.1 + 0.217 * 11 = -2.713
    % -5.1 + 0.217 * 13 = -2.279
    \draw[line width=0.7pt] (-2.713, -0.25) -- (-2.279, -0.25);
    \draw[line width=0.7pt] (-2.713, -0.25) -- (-2.713, -2);
    \node[align=flush left, font=\footnotesize, minimum width=4cm, text width=4cm] at (-0.813, -2.6) {Длина строки, следующей далее.};

    % /lib64/libc.so.6:
    % -5.1 + 0.217 * 14 = -2.062
    % -5.1 + 0.217 * 30 = 1.41
    \draw[line width=0.7pt] (-2.062, -0.25) -- (1.41, -0.25);
    \draw[line width=0.7pt] (-2.062, -0.25) -- (-2.062, -0.5);
    \node[align=flush left, font=\footnotesize, minimum width=3cm, text width=3cm] at (-0.662, -1.3) {Путь, по которому процесс обратился к файлу.};

    % 1005:
    % -5.1 + 0.217 * 31 = 1.627
    % -5.1 + 0.217 * 35 = 2.495
    \draw[line width=0.7pt] (1.627, -0.25) -- (2.495, -0.25);
    \draw[line width=0.7pt] (1.627, -0.25) -- (1.627, -4);
    \node[align=flush left, font=\footnotesize, minimum width=6cm, text width=6cm] at (4.527, -4.6) {pid процесса, производящего доступ к файлу.};

    % 15:
    % -5.1 + 0.217 * 36 = 2.712
    % -5.1 + 0.217 * 38 = 3.146
    \draw[line width=0.7pt] (2.712, -0.25) -- (3.146, -0.25);
    \draw[line width=0.7pt] (2.712, -0.25) -- (2.712, -3);
    \node[align=flush left, font=\footnotesize, minimum width=4cm, text width=4cm] at (4.612, -3.6) {Device number файловой системы.};

    % 646356:
    % -5.1 + 0.217 * 39 = 3.363
    % -5.1 + 0.217 * 45 = 4.665
    \draw[line width=0.7pt] (3.363, -0.25) -- (4.665, -0.25);
    \draw[line width=0.7pt] (3.363, -0.25) -- (3.363, -2);
    \node[align=flush left, font=\footnotesize, minimum width=4cm, text width=4cm] at (5.263, -2.6) {Номер inode прочитанного файла.};

    % 7:
    % -5.1 + 0.217 * 46 = 4.882
    % -5.1 + 0.217 * 47 = 5.099
    \draw[line width=0.7pt] (4.882, -0.25) -- (5.099, -0.25);
    \draw[line width=0.7pt] (4.882, -0.25) -- (4.882, -0.5);
    \node[align=flush left, font=\footnotesize, minimum width=3cm, text width=3cm] at (6.282, -1.3) {Код возврата системного вызова \texttt{open}.};
\end{tikzpicture}